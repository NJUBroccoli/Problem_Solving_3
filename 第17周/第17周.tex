\documentclass[twocolumn]{article}

\usepackage[utf8]{inputenc}
\usepackage{CJKutf8}
\usepackage{CJK}
\usepackage{algorithm}
\usepackage{algorithmic}
\usepackage{amsmath}
\usepackage{amsthm}
\usepackage{amssymb}
\usepackage{newfloat}
\usepackage{setspace}
\usepackage{tikz}
\usepackage{enumerate}
\usepackage{listings}
\usepackage{fancyhdr}
\allowdisplaybreaks[4]
\usetikzlibrary{arrows,graphs}
\usetikzlibrary{graphs}
\usetikzlibrary{graphs.standard}
\newenvironment{SChinese}{
	\CJKfamily{gbsn}
	\CJKtilde
	\CJKnospace}{}
\pagestyle{fancy}
\fancyhead[L]{Problem Solving III}
\begin{document}
	\begin{CJK}{UTF8}{}	
		\begin{SChinese}	
			\title{问题求解(三)第17周作业}
			\author{161220049\quad 黄奕诚}
			\maketitle
			
			\section*{TC Chapter 32}
			\subsection*{32.1-2}
				因为$P$中每一个字符都不同,所以当找到了一个匹配之后,可以直接跳|$P$|位,而不是1位,具体算法如下:
				\begin{algorithm}
					\caption{NAIVE-STRING-MATCHER($T,P$)}
					\begin{algorithmic}[1]
						\STATE $n=T.length$ \\
						\STATE $m=P.length$ \\
						\STATE $k=0$ \\
						\STATE $s=0$ \\
						\WHILE{$s\le n-m$}
						\STATE $i=1$ \\
						\IF{$T[s]==P[1]$}
						\STATE $k=s$ \\
						\STATE $i=0$ \\
						\WHILE{$T[k+i]==P[i]$ and $i<m$}
						\IF{$i==m$}
						\STATE Print "Pattern occurs with shift" $k$ \\
						\ENDIF
						\STATE $i=i+1$ \\
						\ENDWHILE
						\ENDIF
						\STATE $s=s+i$\\
						\ENDWHILE
					\end{algorithmic}
				\end{algorithm}
			\subsection*{32.1-3}
				\begin{proof}
					因为$d\ge2$,并且前$i-1$个字母匹配的概率为$\frac{1}{d^{i-1}}$,所以字母对比较的期望值为\\
					\begin{flalign}
						(n-m+1)\sum_{i=1}^{m}\frac{1}{d^{i-1}} &= (n-m+1)\frac{1-(\frac{1}{d})^m}{1-\frac{1}{d}}\\
						&=(n-m+1)\frac{1-d^{-m}}{1-d^{-1}}\\
						&\le(n-m+1)\frac{1}{1-d^{-1}} \\
						&\le(n-m+1)\frac{1}{1-\frac{1}{2}}\\
						&=2(n-m+1)
					\end{flalign}
				\end{proof}
			\subsection*{32.1-4}
				假设$P$可以表示为$p_1\diamond p_2\diamond\cdots\diamond p_m$.可以用如下算法在$O(mn)$的时间内判断出$T$中是否存在这样的$P$:
				\begin{algorithm}
					\caption{GAP-STRING-MATCHER($T,P$)}
					\begin{algorithmic}[1]
						\STATE $i=1$ \\
						\WHILE{$i\le m$}
						\STATE $exi=0$
						\FOR{$j=i$ \TO $n$}
						\IF{$T[j]==P[i]$}
						\STATE $exi=1$
						\ENDIF
						\ENDFOR
						\IF{$exi==0$}
						\RETURN\FALSE
						\ENDIF
						\STATE $i=i+1$ \\
						\ENDWHILE
						\RETURN\TRUE
					\end{algorithmic}
				\end{algorithm}
			\subsection*{32.2-1}
				考虑到$15\mod11=4,59\mod11=4,92\mod11=4$,所以共有3次伪命中.
			\subsection*{32.2-2}
				\indent
				
				对于相同长度的$k$个模式,只要在Rabin-Karp算法的预处理模块中对每一个模式计算它的值即可,并且对每一个模式执行第10-14行(即加一个for循环),此时渐进运行时间为$O(km(n-m+1))$.
				
				对于不同长度的$k$个模式,只需要在执行第9行的for循环之前更新模式的长度即可,预先可以将每一个模式的长度存储在数组中,并在第9行的for循环外层再套一个for循环,以更新长度.
			\subsection*{32.2-3}
			\indent
			
				假设$P$是$m\times m$的矩阵,$T$是$n\times n$的矩阵.则对于每个$i\in[1,n-m+1],j\in[1,n-m+1]$计算$T$中每个规模为$m\times m$的子矩阵的值,求值时将其看作一维数组,由左到右,由上到下计算即可,并求出模式$P$的值.先比较值,若相等,为了摒除伪命中,则依次比对各个位上的字符,若全等则为一次occur.如此便完成了Rabin-Karp算法的拓展.
			\subsection*{32.2-4}
				\begin{proof}
					假设$A(x)=B(x)$,则有$\sum_{i=0}^{n-1}(a_i-b_i)x^i\equiv0\mod q$.由Exercise 31.4-4及$q$为素数可知这个方程最多有$n-1$个解.从$q$中选中这$n-1$个解的概率为$P=\frac{n-1}{q}<\frac{n-1}{1000n}<\frac{1}{1000}$.所以若$A\neq B$,则最多有千分之一的概率得到$A(x)=B(x)$.若$A=B$,则必然有$A(x)=B(x)$.
				\end{proof}
			\subsection*{32.3-2}
				\begin{tabular}{|c|c|c|c|c|c|}
					\hline
					现状态  & 输入$a$ & 输入$b$ & 现状态 & 输入$a$ & 输入$b$\\\hline
					 0 & 1 & 0 & 11 & 1 & 12\\\hline
					 1 & 1 & 2 & 12 & 3 & 13\\\hline
					 2 & 3 & 0 & 13 & 14 & 0\\\hline
					 3 & 1 & 4 & 14 & 1 & 15\\\hline
					 4 & 3 & 5 & 15 & 16 & 8\\\hline
					 5 & 6 & 0 & 16 & 1 & 17\\\hline
					 6 & 1 & 7 & 17 & 3 & 18\\\hline
					 7 & 3 & 8 & 18 & 19 & 0\\\hline
					 8 & 9 & 0 & 19 & 1 & 20\\\hline
					 9 & 1 & 10 & 20 & 3 & 21\\\hline
					 10 & 11 & 0 & 21 & & \\\hline
				\end{tabular}
				图太懒不画qaq……
			\subsection*{32.3-3}
			\indent
			
				考虑到$P$有非重叠性质,即若$P_k\sqsupseteq P_q$,则$k=0$或$k=q$.于是相应的状态转移具有如下的特征:\\
				\begin{enumerate}[(1)]
					\item 如果当前输入使得状态机返回到除了0状态(初始状态)、1状态以外的任意状态,则已构造的一些后缀是要找的$P$的前缀\\
					\item 除去(1)中情况,若状态返回到0状态,则输入不是pattern的第一个字母\\
					\item 除去(1)(2)中情况,则会返回到1状态,即输入是pattern的第一个字母
				\end{enumerate}
			\subsection*{32.3-5}
			\indent
			
				设$P$可以表示为$p_1\diamond p_2\diamond\cdots\diamond p_m$,思路与32.1-4类似,先寻找$P_1$,找到后再往后找$P_2$,如此往复.此时涉及到状态机,将其修改为:$P_i$状态所能接受的状态不再是pattern中涉及的状态,而只有$P_{i+1}$.如此可以在$O(n)$的时间内在$T$中匹配$P$.
		\end{SChinese}
	\end{CJK}
\end{document}