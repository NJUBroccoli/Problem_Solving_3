\documentclass[twocolumn]{article}

\usepackage[utf8]{inputenc}
\usepackage{CJKutf8}
\usepackage{CJK}
\usepackage{amsmath}
\usepackage{amsthm}
\usepackage{amssymb}
\usepackage{newfloat}
\usepackage{setspace}
\usepackage{tikz}
\usepackage{fancyhdr}
\allowdisplaybreaks[4]
\usetikzlibrary{arrows,graphs}
\newenvironment{SChinese}{%
	\CJKfamily{gbsn}%
	\CJKtilde
	\CJKnospace}{}
\pagestyle{fancy}
\fancyhead[L]{Problem Solving III}
\begin{document}
	\begin{CJK}{UTF8}{}	
		\begin{SChinese}	
			\title{问题求解(三)第5周作业}
			\author{黄奕诚 161220049}
			\maketitle
			
			\section*{26.1-1}
				\begin{proof}
					首先,对于还未进行分割但不满足单向条件的$(u,v)$来说,设两点之间的流为$f(u,v)$,因为添加结点$x$后满足
					$c(u,v)=c(u,x)=c(x,v)$,边的最大容量不变,且$u$到$v$原先的路径除了经过$x$,其他也不变,可得$f'(u,v)=
					f'(u,x)=f'(x,v)$,并且此时满足网络流的构成条件,与原图等价.\\
					其次,对于已经对边$(u,v)$进行分割的图$G'$来说,因为结点$x$入度与出度都为1,故满足$f'(u,x)=f'(x,v)$,则
					将$(u,x),(x,v)$替换成同一条有向边$(u,v)$,则有$f'(u,v)=f'(u,x)=f'(x,v)$,仍保持原先的性质.所以分割前后
					是等价的.
				\end{proof}
			\section*{26.1-2}
				\begin{proof}
					在插入源$s$之前,这个网络流的流可以表示为$f_i=\sum_{i=1}^{n}(\sum_{v\in V}^{}f(s_i,v)-\sum_{v\in V}^{}f(v,s_i))$,
					也即对从每一个源$s_i$出发的流进行求和.在插入$s$之后,满足$c(s,s_i)=\infty$,不妨设$f(s,s_i)=\sum_{v\in V}^{}
					f(s_i,v)-\sum_{v\in V}^{}f(v,s_i)=f_i$(因为肯定小于$\infty$,所以合理),于是有$f_i'=\sum_{i=1}^{n}f_i$,由此可知添加源的
					前后流是等价的.对于添加所有$t_i$的同一后继$t$的情况,证法同理.
				\end{proof}
			\section*{26.1-6}
				将家视为发点$s$,将学校视为收点$t$,将一个路口到另一个路口之间的最大“容量”设为$c(v_i,c_j)=1$,假设存在双向的情况,则补等效点;
				根据流量$f=\sum_{v\in V}^{}f(s,v)-\sum_{v\in V}^{}f(v,s)$,若从家到学校存在至少两条没有重边的路径,则有$f\ge2$,假设恰好
				有一条从$s$到$t$的路径,则必存在一条割边$(u,v)$,由流量守恒可知从$u$出来的流量为1,于是$f=1$,若没有从$s$到$t$的路径,则显然
				$f<1$,故有$f\le1$.因此只需要计算最大流,若最大流不小于2,则满足.
			\section*{26.1-7}
				若$(u,x),(x,v)$是两条有向边,则将$x$分成$x_1$和$x_2$,满足$c(u,x_1)=c(u,x),c(x_2,v)=c(x,v),c(x_1,x_2)=l(x)$即可.由于每个顶点分裂为两个(包括$s$和$t$),故总顶点数为$2|V|$,每个顶点为图增加一条边,故总边数为$|V|+|E|$.
			\section*{26.2-2}
				跨越两个集合的流量为$f(S,T)=11+1+7+4-4=19$,容量为$c(S,T)=16+4+7+4=31$.
			\section*{26.2-6}
				如题26.1-2,只需要在所有$s_i$之前插入一个共同全驱的发点$s$,设$p=\sum_{i=1}^{n}\sum_{v\in V}^{}f(s_i,v)$,则从$s$出发的流量为从$s_i$出发的所有流量的总和;同理,在所有$t_i$之后插入一个共同后继的收点$t$,设$q=\sum_{i=1}^{n}\sum_{v\in V}^{}f(v,t_i)$,则到达$t$的流量为到达$t_i$的所有流量之和.于是就构成了单收发点的网络流图,可以在此求解最大流问题.
			\section*{26.2-8}
				\begin{proof}
					在寻找从$s$到$t$的路径$p$的过程中,路径$p$是简单路径,且所有边的流量、容量都为非负值,故不会包含那些从某顶点$v$指向$s$的有向边,所以这对FORD-FULKERSON算法的正确性并无影响.
				\end{proof}
			\section*{26.2-10}
				将FORD-FULKERSON算法稍作修改:在算法执行过程中,若某条边的流量达到最大容量,则移去该边,且不再生成对应的反向边,因为通过该边的流量不会超过最大流,所以不影响算法正确性.于是增广路径最多有$|E|$条,因为每条增广路径都至少包含一条$c=f$的边.
			\section*{26.2-12}
				\begin{proof}
					由题意知,存在顶点$v$,其指向发点$s$,且$f(v,s)=1$;又$s$是发点(能够通过某路径达到每个顶点).因为存在一个包含$(v,s)$的环,只需要用DFS找到这个环,并将环上所有边的$f$减1,得到$f'$,这个过程共需$O(E)$的运行时间.以下证明修改$f$前后,网络流的性质仍成立:\\
					首先,因为每条边的流为正整数,所以减去1之后仍能满足为非负数;\\
					其次,对于环上的每一个顶点,其流出和流入的流量同时减1,仍满足守恒定律;\\
					再者,在修改$f$的值后不改变最大流,即$|f|=|f'|$.并且有$f'(v,s)=0$.
				\end{proof}
			\section*{26.2-13}
				(此题不是很会……)将每条边的容量乘以$|E|$,再在每个顶点$v$之后添上一条边$(v,x)$,使得$f(v,x)=1$.
			\section*{26.3-3}
				增广路径的上界可以设定为$2*min\{|L|,|R|\}+1$.记$L$中的顶点为$\{l_1,l_2,\cdots,l_{|L|}\}$,$R$中的顶点为$\{r_1,r_2,\cdots,r_{|R|}\}$,记$m=min\{|L|,|R|\}$.则可能的最长增广路径为:$\{s,l_m,r_{m-1},l_{m-1},r_{m-2},l_{m-2},\cdots,l_2,r_1,l_1,r_m,t\}$,其长度即为$2m+1$.
			\section*{26-1}
				\subsection*{(a)}
					本题的构造和26.1-6中两个熊孩子的问题很相似.为每个出发点$s_1,s_2,\cdots,s_m$添加一个共同前驱$s$,并为每个边界点$t_1,t_2,\cdots,c_{4n-4}$添加一个共同后继$t$,且$f(t_i,t)=\infty$,将从$s$出发能到达的非边界非出发点设为该网络流中的中间结点.然后即可通过判断是否有$m$条无重边的增广路径来判断是否能够“逃脱”.
				\subsection*{(b)}
					设对于任意$(u,v)\in E$,有$c(u,v)=1$,由26.1-6题类似地,若该网络流的$f=m$,则可以找到逃脱的路径,若$f<m$,则不满足.$f$不可能大于$m$,因为从$s$出发的流只有$m$.
			\section*{26-2}
				\subsection*{(a)}
					建立图$G'=(V',E')$,其中$V'=\{x_0,x_1,\cdots,x_n
					\}\cup\{y_0,y_1,\cdots,y_n\}$,$E'=\{(x_0,x_i):i\in V\}\cup\{(y_i,y_0):i\in V\}\cup\{(x_i,y_i):(i,j)\in E\}$,并给每条边的容量赋值1,然后运行最大流算法.在算法运行结束后,遍历所有边,将流量为1的边添加到path cover中,则可以得到一条最短的路径覆盖.下列证明它满足“路径覆盖”的要求:\\
					假设存在一个顶点$y_i$,它存在于两条不同的路径上,也即存在$x_j,x_k$,使得$f(x_j,y_i)=f(x_k,y_i)$,但是这样流入$y_i$的流量至少为2,而流出$y_i$的流量为1,与守恒定律矛盾.同理,邻接于$y_i$的顶点只能有1个,否则也与守恒定律矛盾.因此满足“路径覆盖”的要求.
				\subsection*{(b)}
					对于含圈的有向图不奏效.例子:$V=\{1,2,3,4\},E=\{(1,2),(2,3),(3,4),(4,2)\}$此时算法输出结果的最小路径覆盖为1,2,3,4,而在图中由边$(x_1,y_2),(x_2,y_3),(x_3,y_4)$组成了最大流.
			
		\end{SChinese}
	\end{CJK}
\end{document}