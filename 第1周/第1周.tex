\documentclass[twocolumn]{ctexart}

\usepackage{amsmath,amssymb,amsthm}
\usepackage{tikz}
\usepackage{algorithm,algorithmic}

\usetikzlibrary {positioning}
%\usepackage {xcolor}
\definecolor {processblue}{cmyk}{0.96,0,0,0}

\title{问题求解(三)第1周作业}
\author{黄奕诚 161220049}

\begin{document}
	\maketitle
	\section*{TC Chapter 24}
	\subsection*{Exercise 24.1-2}
		\begin{proof}
			由条件知``图$G$不包含从$s$可以到达的权重为负值的环路'',故对于$s$与$V$中除去$s$外的任意一顶点$v$,都存在$\delta(s,v)$。在初始化后$v.d=\infty$随后通过RELAX操作,只有当$v.d>u.d+\omega(u,v)$时,$v.d$会减少(不再为$\infty$),与此同时有$v.\pi=u$,于是$v$具有前驱结点.因此若存在$s$到$v$的路径,则$v$必然执行过RELAX,因此有$v.d<\infty$.
		\end{proof}
	\subsection*{Exercise 24.1-3}
		为方便起见,在改变BELLMAN-FORD算法的同时,微改RELAX算法,使其可以返回一个``$v.d$是否改变''的$bool$值.
	
		\begin{algorithm}
			\caption{RELAX($u,v,w$)}
			\begin{algorithmic}[1]
			\IF{$v.d>u.d+\omega(u,v)$}
			\STATE $v.d=u.d+\omega(u,v)$\
			\STATE $v.\pi=u$\
			\RETURN \TRUE
			\ENDIF
			\RETURN \FALSE
			\end{algorithmic}
		\end{algorithm}
		\begin{algorithm}
		\caption{BELLMAN-FORDE($G,\omega,s$)}
		\begin{algorithmic}[1]
		\STATE INITIALIZE-SINGLE-SOURCE($G,s$)\
		\FOR{ $i=1$ \TO $|G.V|-1$}
			\STATE $flag$=\FALSE\
        	\FOR{each edge$(u,v)\in G.E$}
				\IF{\mbox{RELAX($u,v,w$)==\TRUE\space and $flag$==\FALSE}}
				\STATE $flag$=\TRUE
				\ENDIF	
			\ENDFOR
			\IF{$flag$==\FALSE}
			\RETURN \TRUE
			\ENDIF
        \ENDFOR
        \FOR{each edge$(u,v)\in G.E$}
        \IF{$v.d>u.d+\omega(u,v)$}
        \RETURN \FALSE
        \ENDIF
        \ENDFOR
		\end{algorithmic}
		\end{algorithm}	
	\subsection*{Exercise 24.1-4}
		只需要将第7行换为$v.d=-\infty$即可,并且算法返回的值仍符合原先的要求。算法如下:\\
		\begin{algorithm}
		\caption{BELLMAN-FORD($G,\omega,s$)}
		\begin{algorithmic}[1]
			\STATE INITIALIZE0SINGLE-SOURCE($G,s$)\
			\FOR{$i=1$ \TO $|G.V|-1$}
			\FOR{each edge $(u,v)\in G.E$}
			\STATE RELAX$(u,v,\omega)$\
			\ENDFOR
			\ENDFOR
			\STATE $flag$=\TRUE\
			\FOR{each edge$(u,v)\in G.E$}
			\IF{$v.d>u.d+\omega(u,v)$}
			\STATE $v.d=-\infty$\
			\STATE $flag$=\FALSE\
			\ENDIF
			\ENDFOR
			\RETURN $flag$
		\end{algorithmic}
		\end{algorithm}
	\subsection*{Exercise 24.2-2}
		\begin{proof}
			对于拓扑排序中最后一个结点$u$,假设其有后继结点,则与拓扑排序矛盾,因此不存在$v\in G.Adj[u]$,因此算法第4、5行不会被执行。因此,改变后的算法仍然正确,与原算法等效。
		\end{proof}
	\subsection*{Exercise 24.3-2}
		\begin{center}
			\begin {tikzpicture}[-latex ,auto ,node distance =4 cm and 5cm ,on grid ,
			semithick ,
			state/.style ={ circle ,top color =white , bottom color = processblue!20 ,
				draw,processblue , text=blue , minimum width =1 cm}]
			\node[state] (A) {$0$};
			\node[state] (B) [below =of A] {$-1$};
			\path (A) edge [bend left=30] node {$-1$} (B);
			\path (B) edge [bend left=30] node {$-1$} (A);
		\end{tikzpicture}
		\end{center}
		根据DIJKSTRA算法,得到的下方结点的$d$值为-1,而实际上应该为$-\infty$,因此得到了不正确的结果。
		\begin{proof}
			定理24.6的证明不成立的关键在于:不能满足$\delta(s,y)\le\delta(s,u)$
		\end{proof}
	\subsection*{Exercise 24.3-4}
		\begin{algorithm}
		\caption{CHECK-OUTPUT($G,\omega,s$)}
		\begin{algorithmic}[1]
			\STATE $G'=G$\
			\STATE DAG-SHORTEST-PATHS($G',\omega,s$)\
			\STATE topologically sort the vertices of $G$\
			\FOR{each vertex $v_1$ in $G$, $v_2$ in $G'$, taken in topologically sorted order}
			\IF{$v_1.\pi\neq v_2.\pi$ or $v_1.d\neq v_2.d$}
			\RETURN \FALSE
			\ENDIF
			\ENDFOR
			\RETURN \TRUE
		\end{algorithmic}
		\end{algorithm}
	\subsection*{Exercise 24.3-7}
	该图将每条边$(u,v)\in E$替换成$\omega(u,v)$条具有单位权重的边,于是边数$|E'|=\sum_{(u,v)\in E}^{}\omega(u,v)$,因此结点个数变为$|V'|=\sum_{(u,v)\in E}^{}\omega(u,v)-1$.
	\begin{proof}
		由于广度优先搜索不计权重,每次从灰色结点搜与其相邻的白色结点(可以将所有权重视为1),因此对于$V$中一结点$v_0$,其相邻的$V$中其他结点与其距离越近,便越早搜索到,便越早涂上黑色;这与Dijkstra算法从优先队列中抽取结点的次序是一致的。
	\end{proof}
	\subsection*{Exercise 24.5-2}
	\begin{center}
		\begin {tikzpicture}[-latex ,auto ,node distance =2 cm and 2.5cm ,on grid ,
		semithick ,
		state/.style ={ circle ,top color =white , bottom color = processblue!20 ,
			draw,processblue , text=blue , minimum width =1 cm}]
		\node[state] (S) {$S$};
		\node[state] (A) [below left =of S] {$A$};
		\node[state] (B) [below right =of S] {$B$};
		\path (S) edge node {$1$} (A);
		\path (S) edge node {$1$} (B);
		\path (A) edge node {$0$} (B);
		\end{tikzpicture}
	\end{center}
	如图所示,共有3棵最短路径树,且三条边$(S,A)$,$(S,B)$,$(A,B)$都存在最短路径树包含它,也存在最短路径树不包含它,满足题意。
	\subsection*{Exercise 24.5-5}
	\begin{center}
		\begin {tikzpicture}[-latex ,auto ,node distance =2 cm and 2.5cm ,on grid ,
		semithick ,
		state/.style ={ circle ,top color =white , bottom color = processblue!20 ,
			draw,processblue , text=blue , minimum width =1 cm}]
		\node[state] (S) {$S$};
		\node[state] (A) [below left =of S] {$A$};
		\node[state] (B) [below right =of S] {$B$};
		\path (S) edge [bend right = 20] node {$1$} (A);
		\path (S) edge [bend left = 20] node {$1$} (B);
		\path (A) edge [bend left = 20] node {$0$} (B);
		\path (B) edge [bend left = 20] node {$0$} (A);
		\end{tikzpicture}
	\end{center}
	与此同时,设定$A.\pi=B$,且$B.\pi=A$,于是$G_{\pi}$中形成了一条环路。
	\subsection*{Exercise 24-2}
	\subsubsection*{a.}
		\begin{proof}
			假设盒子$x$嵌套于盒子$y$中,盒子$y$嵌套于盒子$z$中,也即存在排列$\pi$,使得$x_{\pi(1)}<y_1$,$x_{\pi(2)}<y_2$,$\ldots$,$x_{\pi(d)}<y_d$,同时存在排列$\tau$,使得$y_{\tau(1)}<z_1$,$y_{\tau(2)}<z_2$,$\ldots$,$y_{\tau(d)}<z_d$,因此存在一种排列使得$x_{\pi(\tau(1))}<z_1$,$x_{\pi(\tau(2))}<z_2$,$\ldots$,$x_{\pi(\tau(d))}<z_d$.所以嵌套关系是传递的。
		\end{proof}
	\subsubsection*{b.}
		盒子$x$嵌套于盒子$y$ $(*)$的充要条件为:按非递减序排列的$x$各元素在各个位上都比f非递减序排列的$y$各元素小。   $(**)$
		\begin{proof}
			若满足$(**)$,则将非递减序中的$x$与$y$各元素两两配对,并使$y$恢复先前次序,同时根据配对改变$x$的次序,便存在使之成立的一种排列;\\
		    若满足$(*)$,则存在一个排列$\pi$,使得$x_{\pi(1)}<y_1$,$x_{\pi(2)}<y_2$,$\ldots$,$x_{\pi(d)}<y_d$,则$x$中的最小值必小于$y$中的最小值(否则与$(*)$矛盾),同理$x$中第$n$小值小于$y$中第$n$小值,于是得到了$(**)$;
		\end{proof}
		\begin{algorithm}
		\caption{JUDGE-NESTED($x,y,d$)}
		\begin{algorithmic}[1]
			\STATE sort $x,y$ in increasing sequence\
			\FOR{$i=1$ \TO $d$}
			\IF{$x_i>y_i$}
			\RETURN \FALSE
			\ENDIF
			\ENDFOR
			\RETURN \TRUE
		\end{algorithmic}
		\end{algorithm}
		这个算法总时间为$O(d\lg d)$.\\
	c.
		\begin{algorithm}
		\caption{LONGEST-SEQUENCE($G,n,d$)}
		\begin{algorithmic}[1]
			\FOR{each pair($B_i,B_j$) in $G$}
				\IF{JUDGE-NESTED($B_i,B_j,d$)}
					\STATE add $B_j$ to $G.Adj[B_i]$\
				\ENDIF
			\ENDFOR
			\STATE $H=\emptyset$,$dis=-\infty$\
			\FOR{each vertex $s$ in $G$}
				\STATE $H_1=\emptyset$\
				\FOR{each vertex $u\in G.V-\{s\}$}
					\STATE $u.color$=WHITE,$u.d=\infty$,$u.\pi=NIL$\
				\ENDFOR
				\STATE $s.color$=GRAY,$s.d=0$,$s.\pi=NIL$,$Q=\emptyset$\
				\STATE ENQUEUE($Q,s$)\
				\WHILE{$Q\neq\infty$}
					\STATE $u$=DEQUEUE($Q$)\
					\FOR{each $v\in G.Adj[u]$}
						\IF{$v.color==WHITE$}
							\STATE $v.d=u.d+1$, $v.\pi=u$\
							\STATE ENQUEUE($Q,v$)\
						\ENDIF
					\ENDFOR
				\ENDWHILE
				\STATE record $u.\pi$ until $s$ into $H_1$\
			%	\STATE $u.color=BLACK$\
				\IF{$u.d>dis$}
				\STATE $dis=u.d$,$H=H_1$\
				\ENDIF
			\ENDFOR
			\RETURN $H$
		\end{algorithmic}
		\end{algorithm}
	\subsection*{Exercise 24-3}
		\subsubsection*{a.}
			可作出如下转化:
			\begin{displaymath}
				R[i_1,i_2]\bullet R[i_2,i_3]\bullet\cdots\bullet R[i_{k-1},i_k]\bullet R[i_k,i_1]>1 
			\end{displaymath}
			\begin{displaymath}
				\ln(R[i_1,i_2])+\ln(R[i_2,i_3])+\cdots+\ln(R[i_{k-1},i_k])+\ln(R[i_k,i_1])>0
			\end{displaymath}
			\begin{displaymath}
			-\ln(R[i_1,i_2])-\ln(R[i_2,i_3])-\cdots-\ln(R[i_{k-1},i_k])-\ln(R[i_k,i_1])<0
			\end{displaymath}
			我们可以通过BELLMAN-FORD算法来检验是否存在负权重的环,因为,从这个不等式可以看出:图中有负权重的环等价于存在不等式对应的这种套利情况。\\
			由于BELLMAN-FORD算法的运行时间为$O(VE)$,所以该算法的运行时间为$O(VE)$。
		\subsubsection*{b.}
			首先用$a$中的算法来判断是否存在这种的一种序列,如果不存在,无需打印;如果存在,则需要进一步打印其中的一种序列。
			关键在于找出图中的负权重的环。首先调用BELLMAN-FORD算法,同时记录每一个结点的$d$值;然后再relax每个结点足够多次,若某结点的$d$值仍然在减少,则其属于某一个负权重的环。将这些$d$值为$-\infty$的结点标记,找出它们其中的一个连通图,便是这样的一个序列.
			
\end{document}