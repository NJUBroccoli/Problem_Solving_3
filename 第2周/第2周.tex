\documentclass[twocolumn]{article}

\usepackage[utf8]{inputenc}
\usepackage{CJKutf8}
\usepackage{CJK}
\usepackage{amsthm}
\usepackage{amssymb}
\usepackage{amsmath}
\usepackage{newfloat}
\usepackage{setspace}
\usepackage{tikz}
\usepackage{algorithm}
\usepackage{algorithmic}
\usetikzlibrary{arrows,automata}

\newenvironment{SChinese}{%
	\CJKfamily{gbsn}%
	\CJKtilde
	\CJKnospace}{}

\begin{document}
	\begin{CJK}{UTF8}{}	
		\begin{SChinese}	
			\title{问题求解(三)第2周作业}
			\author{黄奕诚 161220049}
			
			\maketitle
			\section*{TC Chapter 25}
			\subsection*{25-1.4}
			\begin{proof}
				欲证该算法中所定义的矩阵乘法是相关的,可证明$(A\cdot B)\cdot C = A\cdot(B\cdot C)$,根据EXTEND-SHORTEST-PATHS的矩阵乘法,可令$S = (A\cdot B)\cdot C,S'=A\cdot(B\cdot C)$.\\
				$s_{i,j}=\sum_{t=1}^{n}(\sum_{k=1}^{n}a_{ik}b{kt})c_{tj}$ \\
				$ = \sum_{t=1}^{n}\sum_{k=1}^{n}a_{ik}b{kt}c_{tj}$ \\
				$ = \sum_{t=1}^{n}(a\cdot b)_{it}\cdot c_{tj}$ \\
				$ = (a\cdot b\cdot c)_{ij}$\\
				同理可证$s_{ij}' = (a\cdot b\cdot c)_{ij}$,由此可知其相关性.
			\end{proof}
			\subsection*{25-1.5}
				不妨联想一下多源最短路径算法的矩阵,这是一个$|V|\times|V|$的矩阵,而对于单源最短路径问题,只需要从中抽取出一行从结点$s$出发的行即可.可以定义一个向量与矩阵相乘的运算:$v_{i+1} = v_iW$,当计算$v_{n-1}$之后停止,则能够计算出从$s$出发的最短路径.\\
				与Bellman-Ford算法的关系:实际上,本节中讨论的算法关键是$v_i$表示最多经过$i$条边,在单源最短路径问题上,这与Bellman-Ford算法逐步进行RELAX操作的循环不变式是吻合的.
			\subsection*{25-1.6}
				\begin{algorithm}[]
				\caption{PREDECESSOR-MATRIX(L,W)}
				\begin{algorithmic}[1]
					\STATE $n = L.rows$\
					\FOR{$i = 1$ \TO $n$}
					\FOR{$j = 1$ \TO $n$}
					\FOR{$k = 1$ \TO $n$}
					\IF{$L[i][j] = L[i][k]+W[k][j]$}
					\STATE $\pi[i][j] = k$\
					\ENDIF
					\ENDFOR
					\ENDFOR
					\ENDFOR
				\end{algorithmic}
				\end{algorithm}
			如此可在$O(n^3)$的时间内计算出前驱矩阵.
			\subsection*{25-1.9}
				思路:在循环执行结束后,再多计算一步$L^{(2m)}$即可,若其与$L^{(2m)}$比较.若相等,则无负权重回路,反之则存在负权重回路.
				\begin{algorithm}[]
					\caption{FASTER-ALL-PAIRS-SHORTEST-PATHS(W)}
					\begin{algorithmic}[1]
						\STATE $n = W.rows$\
						\STATE $L^{(1)} = W$\
						\STATE $m = 1$\
						\WHILE{$m < n-1$}
						\STATE let $L^{(2m)}$ be a new $n\times n$ matrix\
						\STATE $L^{(2m)}$ = EXTEND-SHORTEST-PATHS($L^{(m)},W$)\
						\STATE $m = 2m$\
						\ENDWHILE
						\STATE $L'$ = EXTEND-SHORTEST-PATHS($L^{(m)},W$)\
						\IF {$L' = L$}
						\STATE print(No negative circle)\
						\ELSE
						\STATE print(Have negative circle!)\
						\ENDIF
						\RETURN $L^{(m)}$
					\end{algorithmic}
				\end{algorithm}
			\subsection*{25-1.10}
				可以先在FASTER-ALL-PAIRS-SHORTEST-PATHS(W)中记录下所有存在于负权重的环中的结点,方法为:如果一个结点$v$存在于一个负权重的环中,则会出现在某一个$m$时,有$W_{ii} < 0$,每次遍历一遍所有结点,一旦出现上述情况,则记录进集合.在该算法结束后,通过判断各结点之间的连接情况,来计算负权重环路的长度.
			\subsection*{25-2.2}
				可以模仿CLRS P407的定义,将25.1中的EXTEND-SHORTEST-PATHS中的$l_{ij}' = min(l{ij}',l_{ik}+\omega_{kj})$改为$l_{ij}' = l_{ij}'\vee(v_{ik}\wedge\omega_{kj})$.同时定义:
				\begin{displaymath}
					\omega_{ij} = \{
					\begin{matrix}
					0 & \quad i\neq j\textrm{且}(i,j)\notin E \\
					1 & \quad i = j\textrm{或}(i,j)\in E\\
					\end{matrix}	
				\end{displaymath}
				\\
				然后可以运行FASTER-ALL-PAIRS-SHORTEST-PATHS算法.
			\subsection*{25-2.4}
				\begin{proof}
					该算法相对于FLOYD-WARSHALL算法,去掉了所有上标,以前的版本为$d_{ij}^{(k)} = min(d_{ij}^{(k-1)},d_{ik}^{(k-1)}+d_{kj}^{(k-1)})$.即证明这个递归式执行后不改变右侧各个表达式的值.对于$d_{ij}^{(k)}$来说,由于其不经过$k$,与上标无关,故不改变值.而对于$d_{ik}^{(k-1)}$来说,从$i$到$k$的最短路径相当于从$i$到$k-1$的最短路径,两者没有区别,因而不会改变.所以去掉上标后的算法仍然正确.
				\end{proof}
			\subsection*{25-2.6}
				此题与25.1-10想法类似,主要在算法主体执行完毕后,检查所有的$D_{ii}$(也即对角线上的值),若存在负数,则存在负权重的环.
			\subsection*{25-2.8}
				初始化一个所有元素都为0的矩阵,如果一个结点$i$可以到达$j$,则将$(i,j)$填为1.目的是找到一个$O(VE)$的算法,来完成这个任务.\\
				将所有边的权重赋值为1,有$\delta(i,j) < |E|$,对于一个给定的结点$i$,计算$\delta(i,j)$最多需要$|E|$的时间.再对每一个结点如此遍历,则该算法可以在$O(VE)$的时间内结束.
			\subsection*{25-3.2}
				目的:用来解决非连通图的情况,通过文中给出的新结点$s$,可以顺利地计算出每一个结点的情况(包括两种无穷大).而如果使用图中的结点,则对于无法达到的点无法判断它的情况.
			\subsection*{25-3.3}
				总有$\omega = \widehat{\omega}$.由于所有边的权重全为非负值,则有\[\
				d_{uv} = \delta(u,v) = \widehat{\delta}(u,v)
				\]
				又有\[
				\delta(u,v) = \widehat{\delta}(u,v)+h(v)-h(v) \]
				因此总有$h(v) = h(u)$.\\
				由此可推导出$\omega = \widehat{\omega}$.
			\subsection*{25-2}
			\subsubsection*{a.}
				类比二叉堆,并相应改变三个操作的算法,容易得出INSERT,EXTRACT-MIN,DECREASE-KEY的渐进运行时间分别为$O(\log_dn),O(d\log_dn),O(\log_dn)$.当$d = \varTheta(n^{\alpha})$时,代入式子中,得$O(\frac{1}{\alpha}),O(\frac{n^{\alpha}}{\alpha}),O(\frac{1}{\alpha})$.摊还代价为$O(1),O(\lg n),O(1)$.
			\subsubsection*{b.}
				$d = n^{\epsilon}$,在DIJKSTRA算法中使用d叉堆数据结构,运行时间为$O(\frac{n}{\epsilon}+\frac{2n^{\epsilon+1}}{\epsilon})$\\
				也即$O(V^{1+\epsilon}) = O(E)$.
			\subsubsection*{c.}
				对每一个结点都执行b中的算法,即可计算出所有结点对之间的最短路径.
			\subsubsection*{d.}
				此时包含了权重为负值的边,不可使用DIJKSTRA算法,故不能沿用b,c的算法.这里可以使用JOHNSON算法,由于算法复杂度为$O(VE+V^2\lg V)$,将$|E|=\varTheta(V^{1+\epsilon})$代入,得$O(V^{2+\epsilon}+V^{2+p})$,其中$0<\epsilon\le1$,只要满足$\epsilon > p$即可在$O(VE)$时间内计算结束.
\end{SChinese}
	\end{CJK}
\end{document}