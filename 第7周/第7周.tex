\documentclass[twocolumn]{article}

\usepackage[utf8]{inputenc}
\usepackage{CJKutf8}
\usepackage{CJK}
\usepackage{amsmath}
\usepackage{amsthm}
\usepackage{amssymb}
\usepackage{newfloat}
\usepackage{setspace}
\usepackage{tikz}
\usepackage{fancyhdr}
\allowdisplaybreaks[4]
\usetikzlibrary{arrows,graphs}
\usetikzlibrary{graphs}
\usetikzlibrary{graphs.standard}
\newenvironment{SChinese}{%
	\CJKfamily{gbsn}%
	\CJKtilde
	\CJKnospace}{}
\pagestyle{fancy}
\fancyhead[L]{Problem Solving III}
\begin{document}
	\begin{CJK}{UTF8}{}	
		\begin{SChinese}	
			\title{问题求解(三)第7周作业}
			\author{黄奕诚 161220049}
			\maketitle
			
			\section*{28.1-2}
			\textbf{解:}
				$\left [\begin{matrix}
			4 & -5 & 6 \\
			8 & -6 & 7 \\
			12 & -7 & 12  
			\end{matrix}\right] \newline\newline = \left [\begin{matrix}
			4 & -5 & 6 \\
			2 & 4 & -5 \\
			3 & 8 & -6  
			\end{matrix}\right] = \left [\begin{matrix}
			4 & -5 & 6 \\
			2 & 4 & -5 \\
			3 & 2 & 4  
			\end{matrix}\right] \newline\newline = \left [\begin{matrix}
			1 & 0 & 0 \\
			2 & 1 & 0 \\
			3 & 2 & 1  
			\end{matrix}\right] \left [\begin{matrix}
			4 & -5 & 6 \\
			0 & 4 & -5 \\
			0 & 0 & 4  
			\end{matrix}\right]$\\
			其中$L = \left [\begin{matrix}
			1 & 0 & 0 \\
			2 & 1 & 0 \\
			3 & 2 & 1  
			\end{matrix}\right]$\\
			$U = \left [\begin{matrix}
			4 & -5 & 6 \\
			0 & 4 & -5 \\
			0 & 0 & 4  
			\end{matrix}\right]$
			\section*{28.1-3}
			\textbf{解:}
			$\left [\begin{matrix}
			1 & 5 & 4 \\
			2 & 0 & 3 \\
			5 & 8 & 2  
			\end{matrix}\right] \rightarrow \left [\begin{matrix}
			5 & 8 & 2 \\
			2 & 0 & 3 \\
			1 & 5 & 4  
			\end{matrix}\right] \newline \rightarrow \left [\begin{matrix}
			5 & 8 & 2 \\
			0.4 & -3.2 & 2.2 \\
			0.2 & 3.4 & 3.6  
			\end{matrix}\right] \rightarrow \left [\begin{matrix}
			5 & 8 & 2 \\
			0.2 & 3.4 & 3.6 \\
			0.4 & -3.2 & 2.2  
			\end{matrix}\right] \newline \rightarrow \left [\begin{matrix}
			5 & 8 & 2 \\
			0.2 & 3.4 & 3.6 \\
			0.4 & -\frac{16}{17} & \frac{95}{17}  
			\end{matrix}\right]$\\
			于是可以得到	$P = \left [\begin{matrix}
			0 & 0 & 1 \\
			1 & 0 & 0 \\
			0 & 1 & 0  
			\end{matrix}\right],\newline L = \left [\begin{matrix}
			1 & 0 & 0 \\
			0.2 & 1 & 0 \\
			0.4 & -\frac{16}{17} & 1  
			\end{matrix}\right],
			U = \left [\begin{matrix}
			5 & 8 & 2 \\
			0 & 3.4 & 3.6 \\
			0 & 0 & \frac{95}{17}  
			\end{matrix}\right]$\\
			于是方程可转化为\begin{displaymath}
				\left [\begin{matrix}
				1 & 0 & 0 \\
				0.2 & 1 & 0 \\
				0.4 & -\frac{16}{17} & 1  
				\end{matrix}\right]\left [\begin{matrix}
				y_1\\ y_2 \\ y_3  
				\end{matrix}\right] = \left [\begin{matrix}
				5\\ 12 \\ 9  
				\end{matrix}\right]
			\end{displaymath}
			因此\begin{displaymath}
			y=\left [\begin{matrix}
			5 \\ 11 \\ \frac{295}{17}
			\end{matrix}\right]
			\end{displaymath}
			于是\begin{displaymath}
			\left [\begin{matrix}
			5 & 8 & 2 \\
			0 & 3.4 & 3.6 \\
			0 & 0 & \frac{95}{17}  
			\end{matrix}\right]\left [\begin{matrix}
			x_1\\ x_2 \\ x_3  
			\end{matrix}\right] = \left [\begin{matrix}
			y_1\\ y_2 \\ y_3  
			\end{matrix}\right]
			\end{displaymath}
			解之得:\begin{displaymath}
			x=\left [\begin{matrix}
			-\frac{3}{19} \\ -\frac{1}{19} \\ \frac{59}{19} 
			\end{matrix}\right]
			\end{displaymath}
			\section*{28.1-6}
				\begin{proof}
					对于任意的$n\ge1$,都存在$n\times n$的零矩阵,其可以表示如下:
					\begin{displaymath}
						\left [\begin{matrix}
						0 & 0 & \cdots & 0 \\
						0 & 0 & \cdots & 0 \\
						\vdots & \vdots & \vdots & \vdots \\
						0 & 0 & \cdots & 0  
						\end{matrix}\right] = \left [\begin{matrix}
						1 & 0 & \cdots & 0 \\
						0 & 1 & \cdots & 0 \\
						\vdots & \vdots & \vdots & \vdots \\
						0 & 0 & \cdots & 1  
						\end{matrix}\right] \left [\begin{matrix}
						0 & 0 & \cdots & 0 \\
						0 & 0 & \cdots & 0 \\
						\vdots & \vdots & \vdots & \vdots \\
						0 & 0 & \cdots & 0   
						\end{matrix}\right]
					\end{displaymath}
					左边即为下三角矩阵L,右边即为上三角矩阵U.又因为零矩阵是奇异矩阵,因此得证.
				\end{proof}
			\section*{28.1-7}
				在LU-DECOMPOSITION中,当$k=n$时,有必要执行最外层的for循环迭代.因为若不执行,则$u_{nn}$会呈现原先的值而不是正确的$a_{nn}$\\
				在LUP-DECOMPOSITION中,当$k=n$时,没必要执行最外层的for循环迭代.此时行交换是对最后一行自身的交换,排列矩阵为零矩阵,不影响;再者,第16行的for循环也不会执行。因此这层循环没有影响,故没有必要.
			\section*{28.2-1}
				\begin{proof}
					首先,运行时间为$M(n)$的矩阵乘法可以推出$O(M(n))$的矩阵平方算法。因为对于$B=A^2$,相当于$B=A\cdot A$,也即一切矩阵平方都是矩阵乘法,故这一点容易得到;\\
					其次,欲从运行时间为$S(n)$的矩阵平方算法推出$O(S(n))$的矩阵乘法。对于相乘的矩阵$A$和$B$,不妨构造矩阵\begin{displaymath}
						C = \left [\begin{matrix}
						A & I \\
						B & 0 \\
						\end{matrix}\right]
					\end{displaymath}
					于是\begin{displaymath}
					C^2 = \left [\begin{matrix}
					A^2+B & A \\
					AB & B \\
					\end{matrix}\right]
					\end{displaymath}
					此时矩阵的左下角即为$AB$.由此可知$S(2n)=O(S(n))$,又因为刚才矩阵平方算法的复杂度为$S(n)=O(n^2)$,又因为$S(n)=\Omega(n^2)$,故有$O(n^2)$的矩阵乘法,即$O(S(n))$的矩阵乘法.
				\end{proof}
			\section*{28.2-2}
				\begin{proof}
					不妨设$n=2^k$,将$A$矩阵分块如下:\begin{displaymath}
						A = \left [\begin{matrix}
						A_1 \\ A_2
						\end{matrix}\right]
					\end{displaymath}
					设$A_1 = L_1U_1P_1$,其中$L_1$为$\frac{n}{2}\times\frac{n}{2}$,$U_1$为$\frac{n}{2}\times n$,$P_1$为$n\times n$.进一步矩阵分块,设$U_1=[B|C],A_2P_1^{-1}=[D|F]$,其中$B$与$D$都是$\frac{n}{2}\times\frac{n}{2}$的矩阵.由于$A$为非奇异矩阵,则$B$非奇异,设$G=F-DB^{-1}C$,有\begin{displaymath}
						A = \left [\begin{matrix}
						L_1 & 0 \\
						DB^{-1} & I_{n/2} \\
						\end{matrix}\right]\left [\begin{matrix}
						B & C \\
						0 & G \\
						\end{matrix}\right]P_1
					\end{displaymath}
					再令$G=L_2U_2P_2$,并设$H=\left [\begin{matrix}
					I_{n/2} & 0 \\
					0 & P_2 \\
					\end{matrix}\right]$,故有\begin{displaymath}
						A = \left [\begin{matrix}
						L_1 & 0 \\
						DB^{-1} & L_2 \\
						\end{matrix}\right]\left [\begin{matrix}
						B & CP_2^{-1} \\
						0 & U_2 \\
						\end{matrix}\right]HP_1
					\end{displaymath}
					这便是$A$的一个$LUP$分解。计算过程中计算了子矩阵的LUP分解、矩阵的乘法以及逆矩阵运算,后两者时间上等价,由运行时间的相加关系知LUP分解的运行时间为$O(M(n))$\\
					对于证明的另一半,我并没有找到好的解决方法……
				\end{proof}
			\section*{28.2-3}
				\begin{proof}
					由上题知LUP分解(或是LU分解)的运行时间为$O(M(n))$,则运行这个算法,并沿对角线将每个元素相乘,即是计算该矩阵的行列式,运行时间为$O(M(n))$.\\
			        证$O(D(n))$暂想不出解决方法(如何将求一个行列式的值构造成矩阵的乘法?)
				\end{proof}
			\section*{28.3-1}
				\begin{proof}
					可以进行如下构造:设列向量$e_i$为除了在第$i$行为1,其他位置上都是0的向量;易知行向量$e_i^T$除了在第$i$列为1,其他位置上都是0.于是有\begin{displaymath}
						e_i^TAe_i=\left [\begin{matrix}
						0 & 0 & \cdots & 1 & \cdots & 0
						\end{matrix}\right]\left [\begin{matrix}
						a_{11} & a_{12} & \cdots & a_{1n} \\
						a_{21} & a_{22} & \cdots & a_{2n} \\
						\vdots & \vdots & \vdots & \vdots \\
						a_{n1} & a_{n2} & \cdots & a_{nn}
						\end{matrix}\right]\left [\begin{matrix}
						0 \\ 0 \\ \vdots \\ 1 \\ \vdots \\ 0
						\end{matrix}\right] 
					\end{displaymath}
					前两个矩阵相乘后结果为一个行向量,其中第$j$列都是$a_{ij}$,再与第三个矩阵相乘,可得唯一的数$a_{ii}$.因为$e_i^TAe_i>0$,因此$a_{ii}>0$,这对任意的$i\in [1,n]$都成立.因此对称正定矩阵的对角线上所有元素都为正数.
				\end{proof}
			\section*{28.3-3}
				\begin{proof}
					可以进行如下构造:设列向量$e_i,e_i$为除了在第$i,j$行为1,其他位置上都是0的向量(其中$i\neq j$);于是有\begin{displaymath}
					(e_i-e_j)^TA(e_i-e_j)=
					\end{displaymath}
					\begin{displaymath}
					(e_i-e_j)^T\left [\begin{matrix}
					a_{11} & a_{12} & \cdots & a_{1n} \\
					a_{21} & a_{22} & \cdots & a_{2n} \\
					\vdots & \vdots & \vdots & \vdots \\
					a_{n1} & a_{n2} & \cdots & a_{nn}
					\end{matrix}\right]\left [\begin{matrix}
					0 \\ \cdots \\ -1 \\ \vdots \\ 1 \\ \vdots \\ 0
					\end{matrix}\right] 
					\end{displaymath}
				\end{proof}
				前两个矩阵相乘后结果为一个除了第$j$列为-1,第$i$列为1,其他都为0的行向量,再与第三个矩阵相乘,结果为$a_{ii}+a_{jj}-a_{ij}-a_{ji}=a_{ii}+a_{jj}-2a_{ij}$.假设矩阵各元素中,最大值不在对角线上,则不妨设$a_{ij}$为最大值,此时$a_{ij}\ge a_{ii},a_{ij}\ge a_{jj}$,此时得到$(e_i-e_j)^TA(e_i-e_j)\le0$,这与正定矩阵的定义矛盾.因此最大元素存在于对角线上.
			\section*{28-1}
			\subsection*{a.}
				\begin{displaymath}
					A = \left [\begin{matrix}
					1 & -1 & 0 & 0 & 0 \\
					-1 & 2 & -1 & 0 & 0 \\
					0 & -1 & 2 & -1 & 0 \\
					0 & 0 & -1 & 2 & -1 \\
					0 & 0 & 0 & -1 & 2
					\end{matrix}\right] 
					\end{displaymath}
					\begin{displaymath} 
					= \left [\begin{matrix}
					1 & -1 & 0 & 0 & 0 \\
					-1 & 1 & -1 & 0 & 0 \\
					0 & -1 & 2 & -1 & 0 \\
					0 & 0 & -1 & 2 & -1 \\
					0 & 0 & 0 & -1 & 2
					\end{matrix}\right]
					\end{displaymath}
					\begin{displaymath}
					= \left [\begin{matrix}
					1 & -1 & 0 & 0 & 0 \\
					-1 & 1 & -1 & 0 & 0 \\
					0 & -1 & 1 & -1 & 0 \\
					0 & 0 & -1 & 2 & -1 \\
					0 & 0 & 0 & -1 & 2
					\end{matrix}\right] 
					\end{displaymath}
					\begin{displaymath}
					= \left [\begin{matrix}
					1 & -1 & 0 & 0 & 0 \\
					-1 & 1 & -1 & 0 & 0 \\
					0 & -1 & 1 & -1 & 0 \\
					0 & 0 & -1 & 1 & -1 \\
					0 & 0 & 0 & -1 & 2
					\end{matrix}\right]
					\end{displaymath}
					\begin{displaymath}
					= \left [\begin{matrix}
					1 & -1 & 0 & 0 & 0 \\
					-1 & 1 & -1 & 0 & 0 \\
					0 & -1 & 1 & -1 & 0 \\
					0 & 0 & -1 & 1 & -1 \\
					0 & 0 & 0 & -1 & 1
					\end{matrix}\right] 
				\end{displaymath}
				因此\begin{displaymath}
					L = \left [\begin{matrix}
					1 & 0 & 0 & 0 & 0 \\
					-1 & 1 & 0 & 0 & 0 \\
					0 & -1 & 1 & 0 & 0 \\
					0 & 0 & -1 & 1 & 0 \\
					0 & 0 & 0 & -1 & 1
					\end{matrix}\right] 
				\end{displaymath}
				\begin{displaymath}
				U = \left [\begin{matrix}
				1 & -1 & 0 & 0 & 0 \\
				0 & 1 & -1 & 0 & 0 \\
				0 & 0 & 1 & -1 & 0 \\
				0 & 0 & 0 & 1 & -1 \\
				0 & 0 & 0 & 0 & 1
				\end{matrix}\right] 
				\end{displaymath}
			\subsection*{b.}
				先求$Ly=Pb$,即\begin{displaymath}
					\left [\begin{matrix}
					1 & 0 & 0 & 0 & 0 \\
					-1 & 1 & 0 & 0 & 0 \\
					0 & -1 & 1 & 0 & 0 \\
					0 & 0 & -1 & 1 & 0 \\
					0 & 0 & 0 & -1 & 1
					\end{matrix}\right]\left [\begin{matrix}
						y_1 \\ y_2 \\ y_3 \\ y_4 \\ y_5
					\end{matrix}\right]=\left [\begin{matrix}
					1 \\ 1 \\ 1 \\ 1 \\ 1
					\end{matrix}\right]
				\end{displaymath}
				求得\begin{displaymath}
					y = \left [\begin{matrix}
					1 \\ 2 \\ 3 \\ 4 \\ 5
					\end{matrix}\right]
				\end{displaymath}
				然后$Ux=y$,也即\begin{displaymath}
					\left [\begin{matrix}
					1 & -1 & 0 & 0 & 0 \\
					0 & 1 & -1 & 0 & 0 \\
					0 & 0 & 1 & -1 & 0 \\
					0 & 0 & 0 & 1 & -1 \\
					0 & 0 & 0 & 0 & 1
					\end{matrix}\right]\left [\begin{matrix}
					x_1 \\ x_2 \\ x_3 \\ x_4 \\ x_5
					\end{matrix}\right]=\left [\begin{matrix}
					1 \\ 2 \\ 3 \\ 4 \\ 5
					\end{matrix}\right]
				\end{displaymath}
				因此解为\begin{displaymath}
					x = \left [\begin{matrix}
					15 \\ 14 \\ 12 \\ 9 \\ 5
					\end{matrix}\right]
				\end{displaymath}
			\subsection*{c.}
				将矩阵$A^{-1}$视为5个列向量$x_1,x_2,x_3,x_4,x_5$,则有\begin{displaymath}
					Ax_1=[1,0,0,0,0]^T
				\end{displaymath}
				可以得到\begin{displaymath}
					x_1=[5,4,3,2,1]^T
				\end{displaymath}
				同理可求出$x_2,x_3,x_4,x_5$,最后可求得逆矩阵为\begin{displaymath}
					\left [\begin{matrix}
					5 & 4 & 3 & 2 & 1 \\
					4 & 4 & 3 & 2 & 1 \\
					3 & 3 & 3 & 2 & 1 \\
					2 & 2 & 2 & 2 & 1 \\
					1 & 1 & 1 & 1 & 1
					\end{matrix}\right]
				\end{displaymath}
			\subsection*{d.}
				\begin{proof}
					首先,对三对称矩阵来说,第7-10行只需要交换至多两次以获得最大值,故缩为$O(1)$;其次,第13-15行交换是有限常数次数,也为$O(1)$的运行时间;对于第16-19行,同一列上至多有三个元素,故循环次数为常数次,也为$O(1)$.因此,对于三对称矩阵来说,它的LUP分解运行时间为$O(n)$,易知LU分解的运行时间也为$O(n)$.以求逆矩阵$A^{-1}$为基础的算法,因为需要记录逆矩阵中每一个元素的值,这个过程需要$\Theta(n^2)$的时间,加之算法的其他部分,故在最坏情况下运行时间可以达到$\Omega(n^2)$.
				\end{proof}
			\subsection*{e.}
				\begin{proof}
					由(a)中可知,LUP算法循环体中的各部分运行时间都为$O(1)$,故整体运行时间为$O(n^2)$,在forward和backward的过程中,由于$U$与$L$中每一行每一列同样只有很少的有限个非零元素,在计算$Ly=Pb$以及$Ux=y$对每一个元素$x_i$只需$O(1)$时间,共需要$O(n)$的时间.综上,算法运行时间为$O(n)$.
				\end{proof}
		\end{SChinese}
	\end{CJK}
\end{document}