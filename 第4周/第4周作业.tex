\documentclass[twocolumn]{article}

\usepackage[utf8]{inputenc}
\usepackage{CJKutf8}
\usepackage{CJK}
\usepackage{amsmath}
\usepackage{amsthm}
\usepackage{amssymb}
\usepackage{newfloat}
\usepackage{setspace}
\usepackage{tikz}
\usepackage{fancyhdr}
\allowdisplaybreaks[4]
\usetikzlibrary{arrows,graphs}
\newenvironment{SChinese}{%
	\CJKfamily{gbsn}%
	\CJKtilde
	\CJKnospace}{}
\pagestyle{fancy}
\fancyhead[L]{Problem Solving III}
\begin{document}
	\begin{CJK}{UTF8}{}	
		\begin{SChinese}	
			\title{问题求解(三)第4周作业}
			\author{黄奕诚 161220049}
			\maketitle
			
			\section*{GC Chapter 7}
				\subsection*{7.1}
				\subsubsection*{(a)}
					\begin{proof}
						假设$D$不是强连通图,则存在两个顶点$v_1,v_2$,使得不存在一条从$v_1$到$v_2$的路径.由题意知,图$D-v_1$是强连通图,则必存在一个顶点$v_3$,从$v_3$到$v_2$存在一条路径$P_1$;同理,图$D-v_2$是强连通图,则从$v_1$到$v_3$存在一条路径。由此可知从$v_1$到$v_2$存在一条路径,与假设矛盾。因此,$D$是强连通的。
					\end{proof}
				\subsubsection*{(b)}
					\begin{proof}
						假设存在满足这一性质的四阶定向图$D$.易知三阶定向强连通图是唯一(两两首尾相接),则若要满足题意,则$D$去掉任意一个顶点后的三阶图都必须是该图.设该三阶图的顶点为$v_1,v_2,v_3$,边为$(v_1,v_2),(v_2,v_3),(v_3,v_1)$,则另有一顶点$v_4$,在三个顶点中存在一顶点与$v_4$直接相连,不妨设为$v_1$,则存在$(v_1,v_4)$.若删去$v_1$且仍满足余下三顶点构成三阶定向强连通图,则必存在$(v_3,v_4),(v_4,v_2)$,此时若在$D$中删去$v_3$,剩下的边为$(v_1,v_4),(v_4,v_2),(v_1,v_2)$,不满足三阶强连通图,故矛盾。由此得证。
					\end{proof}
				\subsection*{7.2}
					\begin{proof}
						首先,假设$G$是一个欧拉图,则每个顶点的度数都为偶数,延着一个欧拉环给每一条边定向,且方向与环路径方向一致。由于每次经过一个顶点,都会同时增加一个入度和一个出度,由此可推出对任意的顶点,都有入度等于出度,便能构成一个欧拉定向环,得证;\\
						再者,假设$G$存在欧拉定向环,则对每一个顶点都有入度等于出度 ,则该顶点的总度数必为偶数,将每一条定向边添加一条相反方向的边,便构成一个无向图,且每个顶点度数都为偶数,因此是欧拉图。
					\end{proof}
				\subsection*{7.4}
					\begin{proof}
						假设图$D$是强连通的,则对于其中任意两个顶点$v_1,v_2$,存在路径$P_1$,使得$v_1\rightarrow P_1\rightarrow v_2$,也存在路径$P_2$,使得$v_2\rightarrow P_2\rightarrow v_1$.若$P_1=P_2$,则将所有边的方向反向对其连通性没有影响,依然是强连通;若$P_1\neq P_2$,将所有边反向时,则有$v_1\rightarrow P_2\rightarrow v_2$且$v_2\rightarrow P_1\rightarrow v_1$,因此仍然是强连通图;\\
						充分性同理可证。
					\end{proof}
				\subsection*{7.5}
					\begin{proof}
					首先,若有向图$D$是强连通,设$A,B$的诱导图分别为$G(A),G(B)$.则对$D$中任意两点$v_1,v_2$都有路径使$v_1$连通到$v_2$,或使$v_2$连通到$v_1$.若$v_1,v_2\in G(A)$或$v_1,v_2\in G(B)$,由于两者在同一分量中,必有路径使之相连;若$v_1\in G(A),v_2\in G(B)$,则必存在两个顶点,使得$v_1\rightarrow\cdots\rightarrow v_3\rightarrow v_4\rightarrow\cdots\rightarrow v_2$,其中$v_3\in G(A),v_4\in G(B)$,且$v_1$可能等于$v_3$,$v_2$可能等于$v_4$,则$(v_3,v_4)$即为从$G(A)$到$G(B)$的一个有向边,反之从$G(B)$到$G(A)$亦然\\
					其次,若由割边集分割而成的两个连通分量$G(A)$和$G(B)$,存在由$G(A)$指向$G(B)$的边以及由$G(B)$指向$G(A)$的边,分别设其为$(v_1,v_2)$和$(v_3,v_4)$.对于$D$中的任意两个顶点$(u,v)$,若它们在同一个连通分量中,则显然互相连通,若$u\in G(A),v\in G(B)$,则有路径$u\rightarrow\cdots\rightarrow v_1\rightarrow v_2\rightarrow\cdots\rightarrow v$与$v\rightarrow\cdots\rightarrow v_3\rightarrow v_4\rightarrow\cdots\rightarrow u$,由此可推知$D$是强连通图.
					\end{proof}
				\subsection*{7.9}
					\begin{proof}
						首先,若竞赛图$T$是可迁的,假设$(u,v)$是$T$中的弧,$v$连接到的顶点集为$A$,则根据可迁性,对$A$中每一个顶点$a$,都有弧$(u,a)$,于是可得$u$的出度$od(u)\ge 1+od(v)$,因此每两个顶点的出度不同。\\
						再者,若竞赛图$T$中每两个顶点的出度都不同,设共有n个顶点,它们的出度在集合$\{n-1,n-2,\cdots,1,0\}$中,设$od(v_i)=n-i$,其中$1\le i\le n$.对于任意顶点$v_i$,其邻接顶点为$v_{i+1},v_{i+2},\cdots,v_{n})$,对于$v_{i+1}$,其邻接顶点为$v_{i+2},\cdots,v_{n}$,由此可知若$(v_{i},v_{i+1}\in E)$且$(v_{i+1},v_{i+2})\in E$,则有$(v_i,v_{i+2})\in E$,满足可迁性,因此竞赛图$T$是可迁的.
					\end{proof}
				\subsection*{7.10}
					\begin{proof}
						设$u$到$v$的最短路径为$u->v_1\rightarrow v_2\rightarrow\cdots\rightarrow v_{k-1}\rightarrow v$,则不存在弧$(u,v_2),(u,v_3),\cdots,(u,v_{k-1}),(u,v)$,否则便有更短的路径,因此可得$od(u)\le (n-1)-(k-1)=n-k$,由此知$id(u)=n-1-od(u)\ge k-1$.得证.
					\end{proof}
				\subsection*{7.13}
					\begin{proof}
						对于竞赛图中任意两个顶点$u,v$,要么$(u,v)\in E$,要么$(v,u)\in E$,不妨设$\overrightarrow{d}(u,v)=1,\overrightarrow{d}(u,v)>1$,因此两者不等.
					\end{proof}
				\subsection*{7.14}
					\subsubsection*{(a)}
						\begin{proof}
							若顶点个数$n$为奇数,只要保证所有顶点的出度都为$\frac{n-1}{2}$即可.若$n=1$,则必然成立;若$n=3$,则只要成同向环即可,每个顶点出度为1.对于更多顶点的图,只要保证每个顶点出度为$\frac{n-1}{2}$,如此入度也为$\frac{n-1}{2}$,如此每个球队胜场与负场相同,同样获得第一名.
						\end{proof}
					\subsubsection*{(b)}
						\begin{proof}
							若顶点个数$n$为偶数,假设存在所有球队都获得第一名的结局,则所有球队的出度都相等,且入度也相等.设每个顶点的出度为$o$,入度为$i$,有$o+i=n-1$.由有向图第一定理可知$o\cdot n=\frac{n(n-1)}{2}$,得$o=\frac{n-1}{2}$,而因为$n$为偶数,所以$o$不为整数,而这显然不可能.因此不存在这种情况.
						\end{proof}
				\subsection*{7.15}
					\begin{proof}
						用数学归纳法证明如下:\\
						(1)\quad 当$k=3$时,由定理7.9可知其成立;\\
						(2)\quad 假设当$k=t(3\le t\le n-1)$时,结论成立,也即$T$中含有一个长度为$k$得圈;
						(3)\quad 由上述假设,当$k=t+1$时,设此时长度为$k$的圈内有$v_1,v_2,\cdots,v_k,v_1$,若有一点$u$异于圈内之点,且被圈内一点邻接,也邻接到圈内另一点,则圈可扩充为$v_1,v_2,\cdots,v_i,u,v_{i+1},\cdots,v_k,v_1$,因此存在一个长度为$k+1$的圈.\\
						假设不是上述情况,而是:不在圈内的顶点要么全部邻接到圈上,要么被圈上顶点邻接.设$a$符合前一种情况,$b$符合后一种情况.由于是强连通图,故在圈内存在一点$c$,使得扩充成更大圈$v_1,v_2,\cdots,v_i,b,a,v_{i+1},\cdots,v_k,v_1$且去掉$c$,因此存在一个长度为$k+1$的圈. \\
						(4)\quad 综上所述,结论成立.
					\end{proof}
		\end{SChinese}
	\end{CJK}
\end{document}