\documentclass[twocolumn]{article}

\usepackage[utf8]{inputenc}
\usepackage{CJKutf8}
\usepackage{CJK}
\usepackage{algorithm}
\usepackage{algorithmic}
\usepackage{amsmath}
\usepackage{amsthm}
\usepackage{amssymb}
\usepackage{newfloat}
\usepackage{setspace}
\usepackage{tikz}
\usepackage{fancyhdr}
\allowdisplaybreaks[4]
\usetikzlibrary{arrows,graphs}
\usetikzlibrary{graphs}
\usetikzlibrary{graphs.standard}
\newenvironment{SChinese}{
	\CJKfamily{gbsn}
	\CJKtilde
	\CJKnospace}{}
\pagestyle{fancy}
\fancyhead[L]{Problem Solving III}
\begin{document}
	\begin{CJK}{UTF8}{}	
		\begin{SChinese}	
			\title{问题求解(三)第9周作业}
			\author{黄奕诚 161220049}
			\maketitle
			
			\section*{30.1-2}
				\begin{proof}
					首先,$A(x)=a_0+a_1x^1+a_2x^2+a_3x^3+\cdots+a_{n-1}x^{n-1}$\\
					有$A(x_0)=r$,即$A(x_0)=a_0+a_1x_0^1+a_2x_0^2+a_3x_0^3+\cdots+a_{n-1}x_0^{n-1}=r$\\
					则$A(x)=q(x)(x-x_0)+r=(r-q(0)x_0)+(q(0)-q(1)x_0)x+(q(1)-q(2)x_0)x^2+\cdots+q(n-2)x$\\
					可进行如下构造:\\
					设$n\times n$的矩阵A为:\begin{displaymath}
					A=\left [\begin{matrix}
					1 & -x_0 & 0 & \cdots & 0 \\
					0 & 1 & -x_0 & \cdots & 0 \\
					0 & 0 & 1 & \cdots & 0 \\
					\vdots & \vdots & \vdots & \vdots & \vdots \\
					0 & 0 & 0 & \cdots & 1 
					\end{matrix}\right]
					\end{displaymath}
					\begin{displaymath}
						q = (r,q(0),q(1),\cdots,q(n-2))^T
					\end{displaymath}
					\begin{displaymath}
					a = (a(0),a(1),\cdots,a(n-1))^T
					\end{displaymath}
					只要满足:$Aq=a$.因为矩阵$A$已经是一个上三角矩阵,且稀疏,这个方程通过向后替代的方法可以在线性时间内解出向量$q$,也即求出$r$及$q(0),q(1),\cdots,q(n-2)$
				\end{proof}
			\section*{30.1-4}
				\begin{proof}
					假设$n-1$个点值对可以唯一确定一个阶数为$n$的多项式$P_1$.可以在原有的$n-1$个点值对的基础上添加一个点值对$(x_{n-1},y_{n-1})$,于是根据定理30.1可知这共$n$个点值对可以唯一确定一个$n$阶多项式$P_2$.不妨在原$n-1$个点值对的基础上添加另一个点值对$(x'_{n-1},y'_{n-1})$,它不等同于$(x_{n-1},y_{n-1})$,则可以唯一确定一个$n$阶多项式$P_3$.由于两个点值对的构成不同,则有$P_2\neq P_3$.而在添加新的点值对之前,这两个多项式应当是相同的,此时与“$n-1$个点值对可以唯一确定一个阶数为$n$的多项式$P_1$”矛盾.因此,需要$n$个点值对才可以唯一确定一个阶数为$n$的多项式.
				\end{proof}
			\section*{30.1-5}
				计算矩阵$A$的系数的过程可分为如下步骤:\\
				(1)计算$\prod_{j\neq k}^{x-x_j}\cdot(x-x_k)$,因为共计算了$n$次,所以只需要$O(n)$的时间;\\
				(2)我们得到了$\prod_{j}^{}(x-x_j)$的系数表示,此时需要证明对于每一个$k\prod_ {j-k}^{}(x-x_j)$,都只需要$O(n)$的时间,这点由题30.1-2可直接证得.\\
				(3)此时问题就转化为$\sum_{k}^{}y_k\frac{f_k(x)}{f_k(x_k)}$.计算每一个$f_k(x)$只需要$\Theta(n)$的时间,故所有的$f_k(x)$需要$\Theta(n^2)$的时间.除以$f_k(x_k)$再乘以$y_k$需要$O(n)$的时间,因此这个过程共需要$\Theta(n^2)$的时间.\\
				综上所述,整个计算过程的运行时间为$\Theta(n^2)$.
			\section*{30.2-1}
				\begin{proof}
					\begin{displaymath}
						\omega_n^{n/2}=(e^{2\pi i/n})^{n/2}=e^{\pi i}=-1=e^{2\pi i/2}=\omega_2
					\end{displaymath}
				\end{proof}
			\section*{30.2-4}
				见Algorithm 1\\
				\begin{algorithm}
					\caption{RECURSIVE-FFT-INV(a)}
					\begin{algorithmic}[1]
						\STATE $n = a.length$\
						\IF{ $n == 1$}
						\RETURN $a$ \
						\ENDIF	
						\STATE $\omega_n=e^{2\pi i/n}$\
						\STATE $\omega=\frac{1}{n}$\
						\STATE $a^{[0]}=(a_0,a_2,\cdots,a_{n-2})$\
						\STATE $a^{[1]}=(a_1,a_3,\cdots,a_{n-1})$\
						\STATE $y^{[0]}=$RECURSIVE-FFT-INV$(a^{[0]})$\
						\STATE $y^{[1]}=$RECURSIVE-FFT-INV$(a^{[1]})$\
						\FOR{$k=0$ \TO $n/2-1$}
						\STATE $y_k=y_k^{[0]}+\omega y_k^{[1]}$\
						\STATE $y_{k+(n/2)}=y_k^{[0]}-\omega y_k^{[1]}$\
						\STATE $\omega=\omega\omega_n$\
						\ENDFOR
						\RETURN $y$\	
					\end{algorithmic}
				\end{algorithm}
			
			\section*{30.2-5}
				首先,$(\omega_n^{k+n/3})^3=\omega_n^{3k+n}=(\omega_n^{k})^3\cdot\omega_n^n=(\omega_n^k)^3$.于是设$A^{[i]}=\sum_{j=0}^{n/3-1}a_{i+3j}x^j$,有$A(x)=A^{[0]}(x^3)+xA^{[1]}(x^3)+x^2A^{[2]}(x^3)$。运行时间即为$T(n)=3T(n/3)+\Theta(n)=\Theta(n\lg n)$.
			\section*{30.2-7}
				设当$i=1,2,\cdots,n$时,$P_{i,0}(x)=(x-z_{i-1})$,对于每一个小于等于$\frac{n}{2^k}$的正整数$i$,计算$P_{i,k}=P_{i,k-1}\cdot P_{2i,k-1}$.最终要求解的即为$P_{1,[lg(n)]+1}$,此时得到了一个多项式,$n$表示当给定$n$个为零的点时所需要的计算时间.于是\begin{displaymath}
					T(n)=2T(n/2)+\Theta(nlgn)
				\end{displaymath}
				利用主定理的方法可知运行时间满足$O(nlg^2n)$.
			\section*{30.3-2}
				在题30.2-4,已经写出了RECURSIVE-FFT-INV的算法执行过程,这里只需要仿照FFT改进为ITERATIVE-FFT的过程,将RECURSIVE-FFT-INV的算法改进为ITERATIVE-FFT-INV即可,只需要将ITERATIVE-FFT的第7行改写为$\omega=\frac{1}{n}$即可.
			\section*{30-1}
				\subsection*{a.}
					因为$(a+b)(c+d)=ac+ad+bc+bd$,所以$(ax+b)(cx+d)=acx^2+(ad+bc)x+bd=acx^2+((a+b)(c+d)-ac-bd)x+bd$,此时只需要计算$ac,bd,(a+b)(c+d)$即可,符合题意.
				\subsection*{b.}
					不妨设$n$是2的幂(否则将其归为最近的2的幂,这不影响平均运行时间),设两个多项式分别为$A_1(x)=\sum_{j=0}^{n-1}a_{j,1}x^j$和$A_2(x)=\sum_{j=0}^{n-1}a_{j,2}x^j$\\
					\subsubsection*{方法1}
					设$L_1(x)=\sum_{j=0}^{n/2-1}a_{j,1}x^j,H_1(x)=\sum_{j=n/2}^{n-1}a_{j,1}x^j$,第二个多项式也如此设.显然有$A_1(x)=H_1(x)\cdot x^{n/2}+L_1(x)$,于是\begin{displaymath}
						A_1(x)\cdot A_2(x)=(H_1(x)x^{n/2}+L_1(x))(H_2(x)x^{n/2}+L_2(x))
					\end{displaymath}
					由第$a$小问的方法可知,只需要计算三种式子即可完成这一乘法,有\begin{displaymath}
						f(n)=3f(n/2)+\Theta(n)
					\end{displaymath}
					由主定理可知运行时间为$\Theta(n^{lg3})$.
					\subsubsection*{方法2}
					构造方法为:设$O_i(x)=\sum_{j=0}^{n/2-1}a_{2j+1,i}x^j,E_1(x)=\sum_{j=0}^{n/2-1}a_{2j,i}x^j$.有$A_i(x)=xO_i(x^2)+E_i(x^2),i=1,2$.同理将非$x^i$式子作为参数,由$a$中方法可知运行时间为$\Theta(n^{lg3})$.
				\subsection*{c.}
					对于整数乘法来说,只要把它构造为多项式乘法,再利用前几题的结论便可证明运行时间为$\Theta(n^{lg3})$.设这两个整数为$A_i=\sum_{k=0}^{lg(A_i)}a_{k,i}2^k$,则可设多项式为$f(x,i)=\sum_{k=0}^{lg(A_i)}a_{k,i}x^i$,有$f_i(2)=A_i$,于是便由上两题证得结论.
		\end{SChinese}
	\end{CJK}
\end{document}