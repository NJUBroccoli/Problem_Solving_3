\documentclass[twocolumn]{article}

\usepackage[utf8]{inputenc}
\usepackage{CJKutf8}
\usepackage{CJK}
\usepackage{algorithm}
\usepackage{algorithmic}
\usepackage{amsmath}
\usepackage{amsthm}
\usepackage{amssymb}
\usepackage{newfloat}
\usepackage{setspace}
\usepackage{tikz}
\usepackage{enumerate}
\usepackage{listings}
\usepackage{fancyhdr}
\allowdisplaybreaks[4]
\usetikzlibrary{arrows,graphs}
\usetikzlibrary{graphs}
\usetikzlibrary{graphs.standard}
\newenvironment{SChinese}{
	\CJKfamily{gbsn}
	\CJKtilde
	\CJKnospace}{}
\pagestyle{fancy}
\fancyhead[L]{Problem Solving III}
\begin{document}
	\begin{CJK}{UTF8}{}	
		\begin{SChinese}	
			\title{问题求解(三)第12周作业}
			\author{黄奕诚 161220049}
			\maketitle
			
			\section*{TJ Chapter 2}
				\subsection*{13}
					\begin{proof}
						\begin{enumerate}[(1)]
							\item 首先由第二数学归纳法推第一数学归纳法:由第二数学归纳法知,对于某个整数$n_0$,若$S(n_0)$成立,且可以由$S(n_0),S(n_0+1),\cdots,S(k)$可推出$S(k+1)$,其中$k\ge n_0$,则对任意的$n\ge n_0$,$S(n)$都成立.因为$k$是任取的,当取到$n_0,n_0+1,\cdots$时,可以由$S(n_0)$推得$s(n_0+1)$,由$S(n_0),S(n_0+1)$推得$S(n_0+2)$,$\cdots$,将其中多余的条件删去,便有:$S(n_0)\rightarrow S(n_0+1),S(n_0+1)\rightarrow S(n_0+2),]\cdots$由此便推得第一数学归纳法.
							\item 其次由第一数学归纳法推第二数学归纳法:不妨以良序定理为中间结论,先由第一数学归纳法推良序定理,再由良序定理推第二数学归纳法。对于第一点,由Theorem 2.2已经推出,对于第二点,证明如下:\\
							设集合$S$满足如下条件:$n_0\in S$,若$n_0,n_1,\cdots,k\in S$,则$k+1\in S$.假设$S$不等于$A=\{k|k\ge n_0\}$,则令$T$为其补集,且$T$非空.由良序定理知$T$中有最小值$m\in A$,并且$m\neq n_0$,由$T$的定义知$m-1\in S$,由$S$的定义可推知$m\in S$,这与$m\in T$矛盾.因此可证得第二数学归纳法.
							\item 综上所述,第一与第二数学归纳法等价.
						\end{enumerate}
					\end{proof}
				\subsection*{14}
					\begin{enumerate}[(1)]
						\item 首先证明自然数的良序定理可以推知1是最小的自然数:\begin{proof}
							因为自然数满足良序定理,所以存在最小数,而$N=\{n|n\ge1\}$,有$1\in N$,且对任意的$n_0\in N$,都有$n_0\ge1$,因此1是最小的自然数.
						\end{proof}
						\item 由良序定理证明数学归纳法与题13的第(2)部分相同,令上题的$n_0=1$,即可证明$S=\mathbb{N}$.
					\end{enumerate}
				\subsection*{15}
					\begin{enumerate}[(a)]
						\item 39=14*2+11, 14=11*1+3, 11=3*3+2, 3=2*1+1, 2=1*2+0\\gcd(14,39)=1,且$r=14,s=-5$
						\item 234=165*1+69, 165=69*2+27, 69=27*2+15, 27=15*1+12, 15=12*1+3, 12=3*4+0\\gcd(234,165)=3,且$r=12,s=-17$
						\item 9923=1739*5+1228, 1739=1228*1+511, 1228=511*2+206, 511=206*2+99, 206=99*2+8, 99=8*12+3, 8=3*2+2, 3=2*1+1, 2=1*2+0\\gcd(1739,9923)=1,且$r=3709,s=-650$
						\item 562=471*1+91, 471=91*5+16, 91=16*5+11, 16=11*1+5, 11=5*2+1, 5=1*5+0\\gcd(471,562)=1,且$r=-105,s=88$
						\item 23771=19945*1+3826, 19945=1826*10+1685, 1826=1685*1+141, 1685=141*11+134, 141=134*1+7, 134=7*19+1, 7=1*7+1\\gcd(23771,19945)=1,且$r=881,s=-1050$
						\item 4357=3754*1+603, 3754=603*6+136, 603=136*4+59, 136=59*2+18, 59=18*3+5, 18=5*3+3, 5=3*1+2, 3=2*1+1, 2=1*2+0\\gcd(-4357,3754)=1,且$r=1463,s=1698$
					\end{enumerate}
				\subsection*{16}
					\begin{proof}
						假设$a,b$不互质,设gcd$(a,b)=m$,其中$m>1$且$m\in N$.可得$m|a$且$m|b$,因此$m|(ar+bs)$,故$m|1$,可得$m=1$,而$m>1$,矛盾.因此$a,b$互质.
					\end{proof}
				\subsection*{19}
					\begin{proof}
						因为$xy$是完全平方数,所以$xy=k^2$,其中$k\in\mathbb{Z}$.因为$x,y$互质,所以gcd$(x,y)=1$.若$k=1$,则$x=y=1$满足条件,若$k>1$,由算数基本定理可知,$k=p_1p_2\cdots p_i$,其中$p_1p_2\cdots p_i$都是素数,于是$k^2=p_1^2p_2^2\cdots p_i^2$,也即$xy=p_1^2p_2^2\cdots p_i^2$.假设$x,y$中有不是完全平方数的数,则其必含有因数$p_t$(指数为1),此时另一个数也会含有因数$p_t$,因此它们不互质,矛盾。从而$x,y$都是完全平方数.
					\end{proof}
				\subsection*{22}
					\begin{proof}
						对于$\mathbb{Z}$中的任意元素$m$,都可以表示为带余除法:$m=nq+r$,其中$0\le r\le n-1$,由此可知$m-r\equiv0(\mod n)$,也即$m\equiv r(\mod n)$,所以任意整数都与集合$\{0,1,\cdots,n-1\}$中的某个元素关于$n$同余.
					\end{proof}
				\subsection*{28}
					\begin{proof}
						假设$p$是合数,则$p$可表示为$p=mn$,其中$2\le m,n<p$且$m,n\in\mathbb{N}$.由此有$2^p-1=2^{mn}-1=(2^m)^n-1$.设$s=2^m\ge4$,则原式等于$s^n-1$,有$(s-1)|(s^n-1)$.又因为$s-1\ge3>1$,所以$s^n-1$是合数,即$2^p-1$是合数,与其是素数矛盾,因此$p$是素数.
					\end{proof}
				\subsection*{29}
					\begin{proof}
						设自然数$p=p_1p_2p_3\cdots p_k\cdots+1$,$p_i$为素数,其中$p_1,p_2,\cdots,p_k$是其前$k$个素数,且$p_1=2,p_2=3$,由此有$p=6p_3p_4\cdots+1$.假设$p$不是素数,则存在$p_i(1\le i\le n)$使得$p_i|6p_3p_4\cdots$且$p_i|1$,后者显然不成立,因此$p=6p_3\cdots+1$是素数,结合Theorem 2.7知即形如$6n+1$的素数有无穷多个.
					\end{proof}
				\subsection*{30}
					\begin{proof}
						在上题中将$p_1=2,p_2=3$改为$p_1=p_2=2$,正1改为负1,其余同理可证.
					\end{proof}
				\subsection*{31}
					\begin{proof}
						假设存在整数$p,q$使得$p^2=2q^2$,则有$2|p^2$,因为2是素数,故$2|p$,$p$为偶数.设$p=2k$,得$q^2=2k^2$,可得$q$也为偶数.因为2是素数,原命题等价于$p,q$互质.而因为$p,q$都是偶数,显然不互质,所以矛盾.不存在整数$p,q$使得$p^2=2q^2$.另外,假设$\sqrt{2}$是有理数,即可以表示为$\frac{a}{b}$(gcd$(a,b)=1$).此时$a^2=2b^2$,又前面的证明可知并不存在这样的整数$a,b$.因此$\sqrt{2}$是无理数.
					\end{proof}
				\subsection*{P.E. 1}
					\begin{lstlisting}[language={[ANSI]C}, numberstyle=\tiny,keywordstyle=\color{blue!70},commentstyle=\color{red!50!green!50!blue!50},frame=shadowbox, rulesepcolor=\color{red!20!green!20!blue!20}]
#include <stdio.h>
#include <stdlib.h>
#include <string.h>
#include <math.h>

bool isPrime[100000];
void Sieve(int n)
{
for (int i = 2; i <= sqrt(n); i++)
	if (isPrime[i])
	for (int j = i; j*i<= n; j++)
		isPrime[i*j] = false;
for (int k = 2; k <= n; k++)
	if (isPrime[k])
		printf("%d ", k);
printf("\n");
}
int main()
{
	int N;
	while (scanf("%d", &N) == 1) {
		memset(isPrime, true, 
		  sizeof(isPrime));
		Sieve(N);
	}
	return 0;
}
				\end{lstlisting}
				当$N=120$时,输出为2 3 5 7 11 13 17 19 23 29 31 37 41 43 47 53 59 61 67 71 73 79 83 89 97 101 103 107 109 113.
				\subsection*{P.E. 3}
					\begin{lstlisting}[language={[ANSI]C}, numberstyle=\tiny,keywordstyle=\color{blue!70},commentstyle=\color{red!50!green!50!blue!50},frame=shadowbox, rulesepcolor=\color{red!20!green!20!blue!20}]
#include <stdio.h>
int a,b,x,y;
int gcd()
{
	int d=a;
	if (b){
		d=gcd(b,a%b,y,x);
		y-=(a/b)*x;
	}
	else{
		x=1;
		y=0;
	}
	return d;
}
int main()
{
	while (scanf("%d%d",&a,&b)==2)
	{
		int ans=gcd();
		printf("gcd:%d x:%d y:
		%d\n",ans,x,y);
	}
	return 0;
}

					\end{lstlisting}	
			\section*{CS 2.2}
				\subsection*{2}
					因为$a\cdot133-2m\cdot277=1$,所以$a\cdot133=1+2m\cdot277$,故可以保证$a$有一个模$m$的逆元,即$133(\mod m)$.
				\subsection*{4}
					因为gcd$(31,22)=1$,故存在$a$使得$a\cdot_{31}22=1$且只存在一个这样的$a$,当$a=24$\\
					因为gcd$(10,2)=2\neq1$,故不存在$a$使得$a\cdot_{10}2=1$.
				\subsection*{6}
					因为$a\cdot133-m\cdot277=1$,所以由Theorem 2.15可知$a,m$互质,由此它们只有唯一的公因数,即1.
				\subsection*{8}
					由$k=jq+r$,所以$k=qj+r$,由欧几里得算法可知gcd$(q,k)$=gcd$(r,q)$.
				\subsection*{15}
					两者之间有关系:gcd$(j,k)$是gcd$(r,k)$的因数.当gcd$(r,k)=1$,即$r,k$互质时,有gcd$(j,k)$=gcd$(r,k)$=1.
				\subsection*{16}
					由题意知,$m=-qn-r$且$m=q'n+r',r'=n-r$,因此可得$q'=-q-1$.因为$r'$满足$0\le r'< n$,所以对于任意的整数(可以是非正整数)$m$,总存在整数$q',r'$使得$m=nq'+r'$,其中$0\le r'<n$,由此便由Theorem 2.12推广到了Theorem 2.1
				\subsection*{17}
					计算gcd($F_i,F_{i+1}$)时,在拓展GCD算法中,首先判断两者是否相等.因为斐波那契数列中,相等的元素只有$F_1,F_2$,若是它们则返回gcd=1,$x=1,y=0$.否则需要依次计算$q_i,r_i,k_{i+1}$以及$j_{i+1}$,并利用斐波那契数列$F_{i+2}=F_{i+1}+F_i$的性质进行替换.由于每一个$F_i$都可以表示为$F_1,F_2$的线性组合,所以GCD递归必有出口.最终运算结果为:gcd$(F_i,F_{i+1})=1,x=(-1)^{i-1}F_i,y=(-1)^iF_{i-1}$.
				\subsection*{19}
					$\textrm{gcd}(x,y)*\textrm{lcm}(x,y)=xy$.
					\begin{proof}
						设$a=\textrm{gcd}(x,y),b=\textrm{x,y}$,则有$x=ma,y=na$,其中$m,n$互质.所以$b=mna$,此时$ab=mna^2=(ma)(na)=xy=\textrm{gcd}(x,y)*\textrm{lcm}(x,y)$,得证.
					\end{proof}
		\end{SChinese}
	\end{CJK}
\end{document}