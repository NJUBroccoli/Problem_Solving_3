\documentclass[twocolumn]{article}

\usepackage[utf8]{inputenc}
\usepackage{CJKutf8}
\usepackage{CJK}
\usepackage{algorithm}
\usepackage{algorithmic}
\usepackage{amsmath}
\usepackage{amsthm}
\usepackage{amssymb}
\usepackage{newfloat}
\usepackage{setspace}
\usepackage{tikz}
\usepackage{enumerate}
\usepackage{listings}
\usepackage{fancyhdr}
\allowdisplaybreaks[4]
\usetikzlibrary{arrows,graphs}
\usetikzlibrary{graphs}
\usetikzlibrary{graphs.standard}
\newenvironment{SChinese}{
	\CJKfamily{gbsn}
	\CJKtilde
	\CJKnospace}{}
\pagestyle{fancy}
\fancyhead[L]{Problem Solving III}
\begin{document}
	\begin{CJK}{UTF8}{}	
		\begin{SChinese}	
			\title{问题求解(三)第15周作业}
			\author{黄奕诚 161220049}
			\maketitle
			
			\section*{TJ Chapter 8}
			\subsection*{6}
				\begin{enumerate}[(a)]
					\item $d_{min}=2$,最多可以检1位错,无法纠错;\\
					\item $d_{min}=1$,无法检错或纠错;\\
					\item $d_{min}=1$,无法检错或纠错;\\
					\item $d_{min}=2$,最多可以检1位错,无法纠错;\\
				\end{enumerate}
			\subsection*{7}
				\begin{enumerate}[(a)]
					\item 设$X=(x_1,x_2,x_3,x_4,x_5)^T$,由$HX=0$得到\[x_2=0\]\[x_1+x_3+x_5=0\]
					\[x_1+x_4=0\]
					解得零空间为\begin{displaymath}
						(00000),(00101),(10011),(10110)
					\end{displaymath}
					零空间是$(5,3)$块.由于极大线性无关组的秩为2,所以一个生成矩阵为\begin{displaymath}
					\left [\begin{matrix}
					0 & 1 \\
					0 & 0 \\
					1 & 1 \\
					0 & 1 \\
					1 & 0
					\end{matrix}\right]
					\end{displaymath}
					生成矩阵是不唯一的.\\
					\item 设$X=(x_1,x_2,x_3,x_4,x_5,x_6)^T$,由$HX=0$得到
					\[x_1+x_3=0\]\[x_1+x_2+x_4=0\]\[x_2+x_5=0\]\[x_1+x_2+x_6=0\]
					解得零空间为\begin{displaymath}
						(000000),(010100),(111010),(101101)
					\end{displaymath}
					零空间是$(6,4)$块.由于极大线性无关组的秩为2,所以一个生成矩阵为\begin{displaymath}
					\left [\begin{matrix}
					1 & 1 \\
					1 & 0 \\
					1 & 1 \\
					0 & 1 \\
					1 & 0 \\
					0 & 1
					\end{matrix}\right]
					\end{displaymath}
					生成矩阵是不唯一的.\\
					\item 设$X=(x_1,x_2,x_3,x_4,x_5)^T$,由$HX=0$得到
					\[x_1+x_4+x_5=0\]\[x_2+x_4+x_5=0\]
					解得零空间为\begin{displaymath}
						(00000),(00100),(00011),(00111)
					\end{displaymath}
					\begin{displaymath}
						(11001),(11101),(11010),(11110)
					\end{displaymath}
					零空间是$(5,3)$块.由于极大线性无关组的秩为2,所以一个生成矩阵为\begin{displaymath}
						\left [\begin{matrix}
						1 & 1 & 1 & 1\\
						1 & 1 & 1 & 1\\
						0 & 1 & 0 & 1\\
						0 & 0 & 1 & 1\\
						1 & 1 & 0 & 0
						\end{matrix}\right]
					\end{displaymath}
					生成矩阵不唯一.\\
					\item 设$X=(x_1,x_2,x_3,x_4,x_5,x_6,x_7)^T$,由$HX=0$得到
					\[x_4+x_5+x_6+x_7=0\]
					\[x_2+x_3+x_6+x_7=0\]
					\[x_1+x_3+x_5+x_7=0\]
					\[x_2+x_3+x_6+x_7=0\]
					解得零空间为\[(0000000),(1101001),(0101010),(1000011)\]
					\[(1001100),(0100101),(1100110),(0001111)\]
					\[(1110000),(0011001),(1011010),(0110011)\]
					\[(0111100),(1010101),(0010110),(1111111)\]
					零空间是$(7,3)$块.由于极大线性无关组的秩为4,一个生成矩阵为\begin{displaymath}
					\left [\begin{matrix}
					1 & 0 & 1 & 1\\
					1 & 1 & 0 & 1\\
					0 & 0 & 0 & 1\\
					1 & 1 & 1 & 0\\
					0 & 0 & 1 & 0\\
					0 & 1 & 0 & 0\\
					1 & 0 & 0 & 0\\ 
					\end{matrix}\right]
					\end{displaymath}
					生成矩阵不唯一.
				\end{enumerate}
			\subsection*{8}
				我构建的这样一个编码为\[(10100),(01010),(11001)\],它的$d_{min}=3$,可以检2位错,纠1位错.
			\subsection*{9}
				依次计算$Hx_i(i=1,2,3,4)$,可以得到\begin{displaymath}
					Hx_1=\left [\begin{matrix} 0 \\ 0 \\ 1 \end{matrix}\right]
				\end{displaymath}
				说明第4列有错
				\begin{displaymath}
				Hx_2=\left [\begin{matrix} 1 \\ 1 \\ 0 \end{matrix}\right]
				\end{displaymath}
				出现多位错
				\begin{displaymath}
				Hx_3=\left [\begin{matrix} 1 \\ 1 \\ 0 \end{matrix}\right]
				\end{displaymath}
				出现多位错
				\begin{displaymath}
				Hx_4=\left [\begin{matrix} 1 \\ 1 \\ 0 \end{matrix}\right]
				\end{displaymath}
				出现多位错
			\subsection*{11}
				\begin{enumerate}[(a)]
					\item 是标准奇偶校验矩阵.由\[x_1+x_2=0\]\[x_3=x_4=0\]\[x_1+x_5=0\]得到它的标准生成矩阵为\begin{displaymath}
					\left [\begin{matrix} 1 \\ 1 \\ 0 \\ 0 \\ 1 \end{matrix}\right]
					\end{displaymath}
					因为$d_{min}=3$,所以最多可以检2位错,纠1位错.\\
					\item 是标准奇偶校验矩阵.由\[x_2+x_3=0\]\[x_1+x_2+x_4=0\]\[x_2+x_5=0\]\[x_1+x_2+x_6=0\]得到它的标准生成矩阵为\begin{displaymath}
					\left [\begin{matrix} 1 & 0 \\ 0 & 1 \\ 0 & 1 \\ 1 & 1 \\ 0 & 1 \\ 1 & 1 \end{matrix}\right]
					\end{displaymath}
					因为$d_{min}=3$,所以最多可以检2位错,纠1位错.\\
					\item 是标准奇偶校验矩阵.由\[x_1+x_2+x_3=0\]\[x_1+x_4=0\]得到它的标准生成矩阵为\begin{displaymath}
					\left [\begin{matrix} 1 & 0 \\ 0 & 1 \\ 1 & 1 \\ 1 & 0 \end{matrix}\right]
					\end{displaymath}
					因为$d_{min}=2$,所以最多可以检1位错.\\
					\item 是标准奇偶校验矩阵.由\[x_4=0\]\[x_2+x_3+x_5=0\]\[x_1+x_3+x_6=0\]
					\[x_2+x_3+x_7=0\]得到它的标准生成矩阵为\begin{displaymath}
					\left [\begin{matrix}
					 1 & 0 & 0\\
					 0 & 1 & 0\\
					 0 & 0 & 1\\ 
					 0 & 0 & 0\\
					 0 & 1 & 1\\
					 1 & 0 & 1\\
					 0 & 1 & 1 
					 \end{matrix}\right]
					\end{displaymath}
					因为$d_{min}=2$,所以最多可以检1位错.
				\end{enumerate}
			\subsection*{13}
				\begin{enumerate}[(a)]
					\item \begin{displaymath}
						Hx=\left[\begin{matrix}
						0 \\ 0 \\ 1
						\end{matrix}\right]
					\end{displaymath}
					\item \begin{displaymath}
					Hy=\left[\begin{matrix}
					1 \\ 0 \\ 1
					\end{matrix}\right]
					\end{displaymath}
					\item \begin{displaymath}
					Hz=\left[\begin{matrix}
					1 \\ 1 \\ 1
					\end{matrix}\right]
					\end{displaymath}
					\item \begin{displaymath}
					Hx=\left[\begin{matrix}
					0 \\ 1 \\ 1
					\end{matrix}\right]
					\end{displaymath} 
				\end{enumerate}
			\subsection*{18}
				\begin{proof}
					设$x\in C_i$为非零数.定义一个从非零数到0的映射:$y\mapsto x+y$.对于0,可以表示为两个非零数之和,而且每个$x$对应唯一的$0$,所以若在第$i$个坐标存在非零数,则存在相同数量的0,于是正好有一半是0.若不存在非零数,则全为0.
				\end{proof}
			\subsection*{19}
				\begin{proof}
					设$x\in C$有奇权重,定义一个从奇码字到偶码字的映射:$y\mapsto x+y$.对于任意一个偶码字,都可以表示为两个奇码字之和,并且每个奇码字加上$x$对应唯一的偶码字,因此该映射是双射.所以若存在奇权重的码字,则存在相等数量的偶权重码字,此时正好有一半的偶权重码字.或者不存在奇权重码字,即全为偶权重码字.
				\end{proof}
			\subsection*{21}
				当ASCII码的数量为128时,信息位有7位,所以对于$2^i-1>i+7$,解之得$i\ge4$,所以至少需要4位校验位,于是需要的矩阵规模为$4\times11$;\\
				当ASCII码的数量为256时,信息位为8位,所以对于$2^i-1>i+8$,解之得$i\ge4$,所以至少需要4位校验位,于是需要的矩阵规模为$4\times12$;\\
				若仅需要检错功能,则各需要的校验位为1位,对应的矩阵规模分别为$1\times7$和$1\times8$.
			\subsection*{22}
				三位信息位对应的标准偶校验矩阵为\begin{displaymath}
					\left[\begin{matrix}
					1 & 1 & 1 & 1
					\end{matrix}\right]
				\end{displaymath}
				对于七位信息位,则为\begin{displaymath}
					\left[\begin{matrix}
					1 & 1 & 1 & 1 & 1 & 1 & 1 & 1
					\end{matrix}\right]
				\end{displaymath}
				生成矩阵为\begin{displaymath}
					\left[\begin{matrix}
					1 & 0 & 0 & 0 & 0 & 0 & 0\\
					0 & 1 & 0 & 0 & 0 & 0 & 0\\
					0 & 0 & 1 & 0 & 0 & 0 & 0\\
					\vdots & \vdots & \vdots & \vdots & \vdots & \vdots & \vdots\\
					0 & 0 & 0 & 0 & 0 & 0 & 1\\
					1 & 1 & 1 & 1 & 1 & 1 & 1
					\end{matrix}\right]
				\end{displaymath}
			\subsection*{23}
				设需要$n$位校验位,则$2^n-1\ge n+20$,得到$n\ge5$,所以需要5位校验位来实现20位信息位的纠错.又由$2^n-1\ge n+32$,得到$n\ge6$,所以需要6位校验位来实现32位信息位的纠错.
		
		\end{SChinese}
	\end{CJK}
\end{document}